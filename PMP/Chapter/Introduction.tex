\chapter{Introduction}
\label{chap:intro}
\pagenumbering{arabic}
\section{Purpose of document}
   This document provides details about how the development team plans, implements, and monitors the development of the JJFresh online store. 

   According to the project requirements specification, this document describes in detail the scope of the project and possible risks, and how to control it. The technologies that will be used throughout the development process are defined by the development team, while the project plan containing the development team's size and project needs would be defined by the Scrum master.

   This document will serve for all parties involved in the entire development cycle, such as Scrum Master, Product Owner and developers.

\section{Audience of document}
\label{sec:audienceOfDocument}
This document describes in detail the business value, requirements, scope, risks, and implementation plan of the entire project, which will be applied as the only standard throughout the project development phase. It also means that if any conflicts of opinion occur during the project development stage, they are measured and resolved using this document. Both internal and external stakeholders of the project will benefit from it.

First, for the student team who are also development members, they need to use the document frequently. For example, the project plan stipulates what the development team should complete in when. 

Secondly, for Jess and James, they can use this document to understand the development process and methods of the project, as well as the economic value that the project will bring. 

Finally, the teaching team, mainly Rajesh Chittor Sundaram, can use this document to supervise the development team so that they can complete the task efficiently within the stipulated time.

\section{Evolution of document}
\label{sec:evolutionOfDocument}
All of the people in the student team participated in the preparation of the document. Yicun Tian is responsible for writing the Executive Summary, Introduction (Chapter \ref{chap:intro}), Roles and Responsibilities (Section \ref{sec:rolesResponsibilities}), Communication Plan (Section \ref{sec:communicationPlan}), and Risk management (Section \ref{sec:riskManagement}). Hongzhang Li is the person who wrote the Project Information part (Chapter \ref{projectInformation}). Pin Wang and Chongjing Zhang, as developers, are responsible for defining the technology used to implement the project (Section \ref{sec:technology}). Zhangfeng Qiu, who served as Scrum Master, provide detail about the Project Planning (Section \ref{sec:projectPlanning}) and proofread all content in the document. The evolution of document is shown in Table \ref{tab:evolutionOfDocument}.
\clearpage
\begin{tabularx}{0.95\linewidth}{%
  >{\raggedright\arraybackslash}p{0.1\linewidth}%
  >{\raggedright\arraybackslash}p{0.17\linewidth}%
  >{\raggedright\arraybackslash}p{0.13\linewidth}%
  >{\raggedright\arraybackslash}p{0.40\linewidth}
  }
  \toprule
  Version & Created by & Date created & Comments\\
  \midrule
  Version 1.0
  & Yicun Tian,\newline Hongkang Li,\newline Pin Wang,\newline Chongjing Zhang,\newline Zhangfeng Qiu
  & April 27, 2020
  & In this version, the student team cooperated together to complete Chapter \ref{chap:intro} to Chapter \ref{chap:projectGovernance} of the Project Management Plan, including the definition of key stakeholders, Scope and SDLC of the project, evaluation of business value, constraints, the definition of people's role in the team, communication plan, risk management, definition of the technology used in the project and project planning.
  \\
  \midrule
  Version 1.1
  & Yicun Tian,\newline Hongkang Li,\newline Pin Wang,\newline Chongjing Zhang,\newline Zhangfeng Qiu
  & May 18, 2020
  & In this version, the document is updated based on Rajesh Chittor's feedback and the actual situation of the project. Section \ref{sec:ps1} is now included in the document. The detail of the update history is shown in Section \ref{sec:v1-1Up}.
  \\
  \midrule
  Version 1.2
  & Yicun Tian,\newline Hongkang Li,\newline Pin Wang,\newline Chongjing Zhang,\newline Zhangfeng Qiu
  & May 28. 2020
  & In this version, the document is updated to provide detail about the progress of the project status. Section \ref{sec:ps2} is now included in the document. The detail of the update history is shown in Section \ref{sec:v1-2Up}.
  % \\
  % \midrule
  % Version 1.3
  % & 
  % & 6.5
  % & 
  \\
  \bottomrule
  \\
  \caption{Evolution of document}  
  \label{tab:evolutionOfDocument}
\end{tabularx}

\subsection{Version 1.1 update history}
\label{sec:v1-1Up}
  \begin{enumerate}
    \item Rewrite the Executive summary. Remove lengthy expressions.
    \item Provide more detail about the business value of the project in the Executive summary.
    \item Fix the cost estimation problem in the Executive summary.
    \item Redesign all the table in the document. Increase readability and change the date format.
    \item Provide names of the people in the teaching team and the student team.
    \item Move the origin Section \ref{sec:audienceOfDocument} to Section \ref{tab:evolutionOfDocument}. Mention who the document will serve in Section \ref{sec:audienceOfDocument}
    \item Add suppliers into key stackholders in Section \ref{sec:keyStakeholders} and provoid benefit analysis for them in Section \ref{sec:businessValue}.
    \item Listed all requirement separately in Section \ref{sec:scope}.
    \item Fix mistakes about the advantage of Scrum in Section \ref{sec:deliveryApproach}.
    \item Change the roles in Section \ref{sec:rolesResponsibilities}. Now, Yicun Tian and Hongkang Li would play the role of Product owner and the teaching team would play the role of subject matter expert.
    \item Provide detail about the virtual meeting rooms in Section \ref{sec:communicationPlan}.
    \item Provide specific time for communication plan in Section \ref{sec:communicationPlan}.
    \item Add Emergency Meeting and Daily communication into the communication Matrix in Table \ref{tab:communicationMatrix}.
    \item Change the probability of Risk 1 in Section \ref{sec:riskImpactAnalysisTable}. Improve the risk triggers of risk 1 and risk 4 to make a more comprehensive risk impact analysis.
	  \item Add the current project timelines and constraints as input for choosing to use \textit{Wix} in Section \ref{sec:technology}.
    \item Add reference for Bang-for-the-Buck to the footnote.
    \item Redefine the value point and story point in the product backlog and provide milestone definition in Section \ref{sec:projectPlanning}. Remove tasks definition in Table \ref{tab:productBacklog}.  
    \item Re-estimate the velocity of the development team based on the first sprint feedback.
    \item Update the Second Sprint Plan in Section \ref{theSecondSprintPlan}.
    \item Provide detail about the burndown chart, meeting record and product artefacts in Chapter \ref{chap:pe}.
    \item Provide detail about the product. The online store website can be accessed now on \textit{https://pinwang4.wixsite.com/website}.
    \item Provide Risk Monitoring and Control in Section \ref{sub:riskMonitoringandControl}.
  \end{enumerate}

\subsection{Version 1.2 update history}
\label{sec:v1-2Up}{}
  \begin{enumerate}
    \item Fix a date mistake in Executive summary.
    \item Add the second Sprint Review and Sprint Rethospective to Agenda in Section \ref{sec:ps2}.
    \item Add the second Sprint Review and Sprint Rethospective to Meeting Minutes in Appendix A.
    \item Provide the detail of the second Sprint Review and Sprint Retrospective in Section \ref{sec:ps2}.
    \item Provide the second sprint review and sprint retrospective in Section \ref{sec:ps2}.
    \item Provide User Case Diagram in Section \ref{sec:ps2}
    \item Update Risk Monitoring and Control in Section \ref{sub:riskMonitoringandControl2}. 
  \end{enumerate}