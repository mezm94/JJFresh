\documentclass{report}
\usepackage[utf8]{inputenc}
\usepackage[english]{babel}
\usepackage[]{graphicx}
\graphicspath{ {Figures/} }
% Set page margins
\usepackage[top=100pt,bottom=100pt,left=68pt,right=66pt]{geometry}
% Changes the style of chapter headings
\usepackage{titlesec}
\titleformat{\chapter}
   {\normalfont\LARGE\bfseries}{\thechapter.}{1em}{}
% Change distance between chapter header and text
\titlespacing{\chapter}{0pt}{50pt}{2\baselineskip}

\pagenumbering{roman}

\begin{document}
% Title
\begin{titlepage}
	\clearpage\thispagestyle{empty}
	\centering
	\vspace{1cm}

	% Titles
	% Information about the University
	{\normalsize The University of Melbourne \\ 
		School of Computing and Information Systems \\
		SWEN90016 Software Processes and Management\\
		Semester 1 –- 2020 \par}
		\vspace{3cm}
	{\Huge \textbf{JJFresh -- Project Management Plan}} \\
	%\vspace{1cm}
	%{\large \textbf{xxxxx} \par}
	\vspace{4cm}
	{\normalsize Zhaofeng Qiu 1101584 \\ % \\ specifies a new line
	             Chongjing Zhang 1055520 \\
	             Pin Wang 1056745\par}
	\vspace{4cm}
    
    \centering \includegraphics[scale=0.1]{logo.pdf}
    
    \vspace{0.5cm}
		
	% Set the date
	{\normalsize \today \par}
	\pagebreak
\end{titlepage}

\chapter*{Executive Summary}\label{sec:ESummary}
[\textit{Give your stakeholders a concise preview of the project’s plan, purpose and approach. Consolidate the main points of the document to explain why the project is being undertaken, who will be responsible for implementing it, how much it is likely to cost, the desired outcomes and benefits it is likely to produce, and how long it will take to complete. An executive summary should be organised according to the sequence of information presented in the document. Use plain English and ensure all acronyms are fully expanded out the first time they are used. Keep the executive summary as succinct as possible and contained to a single page.}]
\addcontentsline{toc}{chapter}{Executive Summary}
\pagebreak
\clearpage

\tableofcontents
\pagebreak

\chapter{Introduction}
\pagenumbering{arabic}
\section{Purpose of document}
\section{Audience of document}
\section{Evolution of document}

\chapter{Project Information}
\section{Key Stakeholders}
From the project brief identify the key stakeholders for the project.

\section{Scope}
\subsection{What is in-scope?}
Detail the scope of the project. The execution of the entire project starts with a clear and complete scope definition. Every other element of project planning will relate to scope and to the deliverables listed below. Clearly state what requirements your team is planning to deliver in the project.
\subsection{What is out-of-scope?}
It’s equally important to list what the project team isn’t responsible for delivering. This section provides the project team with the opportunity to clearly indicate what is not in scope of the project where there may be any doubt or confusion.
\section{Delivery approach / SDLC - Agile}
\section{Business Value (Financial \& Non-Financial Benefits)}
\section{Constraints}

\chapter{Project Governance}
\section{Roles and Responsibilities}
\section{Communication Plan}
\section{Risk Management}
\section{Technology}
\section{Project Planning}
\section{}

\chapter{Project Execution, Monitoring and Control}
\section{Project Status: Friday Week 9}
\subsection{Process Related Artefacts}
\subsection{Product Related Artefacts}
\subsection{Risk Monitoring and Control}
\section{Project Status: Friday week 10}
\subsection{Process Related Artefacts}
\subsection{Product Related Artefacts}
\subsection{Risk Monitoring and Control}
\section{Project Status: Friday week 10}
\subsection{Process Related Artefacts}
\subsection{Product Related Artefacts}
\subsection{Risk Monitoring and Control}



\end{document}
